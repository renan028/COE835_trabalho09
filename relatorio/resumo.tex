%---------------------------------------------------------------------
\section{Resumo das equa��es}

Abaixo, resumimos algumas das principais equa��es utilizadas no m�todo.

\vspace{5mm}

\noindent

\equacao{Planta}
  {P(s) = \frac{Y(s)}{U(s)} = \frac{k_p}{s^2 + a_{1} s + a_{0}}}

\equacao{Modelo}
  {P_m(s) = \frac{Y_m(s)}{R(s)} = \frac{k_m}{s^2 + a_{m_1} s + a_{m_0}}}

\equacao{Realiza��o do erros}
  {\begin{cases}
    \dot{\epsilon}_1 &= -a_{m_1} \epsilon_1 + \epsilon_2 \\
    \dot{\epsilon}_2 &= -a_{m_0} \epsilon_1 - k_m r + k_m F^T \Psi \\
    z_1 &= e_0 = y - y_m = \epsilon_1
   \end{cases}
}

\equacao{Filtros}
{
\begin{cases}
\dot{\xi} &= N\xi - (a_{m_0} + a_{m_1}N + N^2)e_0 - k_m r\\
\dot{\Omega}^T &= N\Omega^T + k_m F^T\\
\end{cases}
}

\equacao{Regressores}
  {\Omega^T = \mat{v_0 & \Xi^T}\,, \qquad \bar{\Omega}^T = \mat{0 & \Xi^T} \,, \qquad \Xi^T = \mat{v_1 & v_2 & v_3}}
  
\equacao{Fun��o estabilizante}
  { \bar{\alpha} = -c_1 z_1 + \frac{d_1}{N}z_1 + (a_{m_1} + N) z_1 - \xi - \bar{\Omega}^T \hat{\Psi} \,, \qquad \alpha = \hat{\rho} \bar{\alpha}}

\equacao{Adapta��o}
  { \dot{\hat{\rho}} &= -\gamma \text{sign}(k)\bar{\alpha} z_1 \,, \qquad k = k_p/k_m}

\equacao{Fun��o de sintonia 1}
  { \tau_1 &= (\Omega - \hat{\rho} \bar{\alpha} e_1)z_1 }
    
\equacao{Erro $z_2$}
{z_2 = v_0 - \alpha}

\equacao{Sinal auxiliar}
{\beta = -Nv_0 + \frac{\partial\alpha}{\partial z_1} \left(\xi + \Omega^T \hat{\Psi} - a_{m_1} z_1 -Nz_1 \right) + \frac{\partial\alpha}{\partial \hat{\rho}} \dot{\hat{\rho}} + \frac{\partial\alpha}{\partial \xi}\left(N\xi -cz_1 -k_mr \right) + \frac{\partial\alpha}{\partial \bar{\Omega}} \left(N\bar{\Omega} + k_m \mat{0 & -\omega_r^T}^T \right)}

\equacao{Derivadas parciais de $\alpha$}
{
\begin{cases}
 \partial\alpha/\partial z_1 &= \hat{\rho}\left(-c_1 +\frac{d_1}{N} +a_{m_1} +N\right) \\
\partial\alpha/\partial \hat{\rho} &= \bar{\alpha} \\
\partial\alpha/\partial \xi &= -\hat{\rho} \\
\partial\alpha/\partial \hat{\Psi} &= -\hat{\rho}\bar{\Omega}^T \\
\partial\alpha/\partial \bar{\Omega} &= -\hat{\rho}\hat{\Psi}^T
\end{cases}
}

\equacao{Fun��o de sintonia 2}
  { \tau_2 = \tau_1 - \frac{\partial\alpha}{\partial z_1} \Omega z_2 }

\equacao{Adapta��o}
  { \dot{\hat{\Psi}} &= \Gamma \tau_2}
  
\equacao{Sinal de controle}
  { u = \frac{1}{k_m}\left(-c_2z_2 - \hat{k}z_1 + \beta + \frac{\partial\alpha}{\partial \hat{\Psi}} \dot{\hat{\Psi}} + \frac{d_2}{N}\left(\frac{\partial\alpha}{\partial z_1}\right)^2z_2\right)}
  
\textbf{Hip�teses:}
%
\begin{enumerate}
 \item O sinal do ganho de alta frequ�ncia da planta, $k_p$, � conhecido;
 \item O sinal de refer�ncia $r(t)$ � limitado;
 \item $N<0 \,, \quad c_1, c_2, d_1, d_2, \gamma, \Gamma > 0$\,.
\end{enumerate}

