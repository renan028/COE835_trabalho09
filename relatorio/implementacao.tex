\section{Implementa��o}

Para a implementa��o, s� � preciso computar os filtros:
%
\begin{align}
\dot{\lambda} &= N\lambda + u \\
\dot{\eta} &= N\eta + y
\end{align}

Temos ainda que $\alpha =
\hat{\rho}(-c_1z_1-d_1z_1-\xi-\bar{\omega}^\intercal\hat{\theta} + Ny)$. Logo:
\begin{align}
\frac{\partial \alpha_1}{\partial y} &= -\hat{\rho}(c_1 + d_1 +
\mat{0 & -e_1^\intercal}\hat{\theta} - N)\\
%
\frac{\partial \alpha_1}{\partial \eta} &=
\hat{\rho}\left(N^2 +
\mat{0 & N & 1}\hat{\theta}\right) \\
%
\frac{\partial \alpha_1}{\partial y_r} &= \hat{\rho}(c_1 + d_1)\\
\frac{\partial \alpha_1}{\partial \hat{\theta}} &= - \hat{\rho}\bar{\omega}^\intercal \\
\frac{\partial \alpha_1}{\partial \hat{\rho}} &= -(c_1+d_1)(y-y_r) +
N^2\eta -\bar{\omega}^\intercal \hat{\theta} + Ny
\end{align}

Transcrevemos abaixo a fun��o que computa o Backstepping com observador de ordem reduzida:

\lstinputlisting[style=myMatlab]{../matlab/backstepping_red.m} 